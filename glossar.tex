\newglossaryentry{Bewegungsangebote}{name={Bewegungsangebote},description={Unter Bewegungsangeboten werden, Bewegungsm�glichkeiten verstanden, die vom p�dagogischen Fachpersonal gestellten werden. Hierunter fallen r�umliche Gegebenheiten, sowie die Materialien die die Kinder nutzen k�nnen. Die Kinder haben hier unter p�dagogisch Aufsicht, die M�glichkeit frei zu spielen. Freispiel in vorbereiteter Umgebung.}}

\newglossaryentry{Bewegungsspiele}{name={Bewegungsspiele},description={Als Bewegungsspiele werden Spiel bezeichnet, die Bewegungst�tigkeiten von Kindern beschreibt, die sich aus Spielsituationen ergeben und meist selbstgesteuert sind.}}

\newglossaryentry{Egozentrismus}{name={Egozentrismus}, description={Ist die Tendenz, die Welt aus der eigenen Perspektive zu sehen.}}

\newglossaryentry{Empathie}{name={Empathie}, description={Als Empathie oder auch Einf�hlungsverm�gen wird die F�higkeit verstanden, Gef�hle bei anderen wahrzunehmen und sich in Gef�hlslage der andere Person hineinzuversetzen. Des Weiteren setzt es voraus, dass die eigenen Gef�hle von denen des anderen unterschieden werden. Empathie ist nicht zu verwechseln mit Gef�hlsansteckung oder Mitleid. Die F�higkeit empathisch zu denken, handeln und zu reagieren, k�nnen Kinder erst ausbilden, wenn sich das Selbstkonzept herausgebildet hat. Dieses erm�glicht die getrennte Wahrnehmung von dem \enquote{ich} und dem \enquote{anderen}.}}

\newglossaryentry{Freispiel}{name={Freispiel}, description={Das Freispiel ist ein wichtiger Bestandteil in der Tagesgestaltung im Kindergarten oder in der Kindertagesst�tte. Darunter wird verstanden,  Kindern die M�glichkeit zu bieten, w�hrend einer bestimmten Zeit, Spiele frei zu entwickeln und zu gestalten.}}

\newglossaryentry{Freundschaften}{name={Freundschaften}, description={Merkmal von Freundschaft ist die Freiwilligkeit, enge Gegegseitig positiv gesinnte Beziehung.}}

\newglossaryentry{Gleichaltrigen}{name={Gleichaltrigen}, description={siehe Peers}}

\newglossaryentry{Mitleid}{name={Mitleid}, description={Mitleid bezieht sich auf auf das Hineinversetzen in negative Gef�hle oder die Lebenslage einer anderen Person und dr�ckt sich in Anteilnahme, Kummer und Sorge aus.}}

\newglossaryentry{Peers}{name={Peers}, description={Peer sind Menschen von etwa gleichem Alter und Status.}}

\newglossaryentry{Prosoziales verhalten}{name={Prosoziales verhalten}, description={Das ist ein freiwilliges Verhalten, von denen andere einen positiven Effekt haben, wie zum Beispiel, jemanden zu tr�sten, etwas zu teilen oder zu helfen.}}

\newglossaryentry{soziometrischen Status}{name={soziometrischen Status}, description={Mit dem soziometrischer Status wird gemessen wie sehr ein Kind von seinen Peers als Gesamtgruppe gemocht wird.}}

\newglossaryentry{Selbstbild}{name={Selbstbild}, description={Das Selbstbild beschreibt das Bild, das sich ein Kind von seiner Person macht.}}

\newglossaryentry{Gefhlsansteckung}{name={Gef�hlsansteckung}, description={Der Begriff Gef�hlsansteckung, bezieht sich auf den emotionalen Zustand, in dem die Person durch die Identifikation sich in die gleiche emotionale Lage bring. Dieses Verhalten kann man schon bei S�uglingen beobachten. Wenn sie ein anderes Kind weinen h�ren, tun sie es ihm gleich.}}
