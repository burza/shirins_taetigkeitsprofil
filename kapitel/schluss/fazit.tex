%---------------------------------------------------------------------------------------------------
% Zusammenfassung
%---------------------------------------------------------------------------------------------------
% \newpage
%%\part{Schluss}
\chapter{Fazit}
  \begin{flushleft}
    Die Zielsetzung dieser Arbeit war es, die besondere Chance von Spiel und Bewegung hinsichtlich der F�rderung sozial-emotionaler Kompetenzen, zu beleuchten. Zu diesem Zweck wurde erst gekl�rt, was emotionale und soziale Kompetenzen sind. Im weiteren Verlauf wurde dann darauf eingegangen, welchen Einfluss sozial-emotionale Kompetenzen auf das Leben eines jeden Menschen haben und welche Rolle Konflikte hierbei spielen. Dem folgte die begriffliche  Auseinandersetzung von Spiel und Bewegung. Zudem wurde der Erwerb sozial-emotionale Kompetenzen durch Spiel und Bewegung er�rtert. Anhand der daraus sich erschlie�enden Erkenntnisse, wurde noch beleuchtet, welchen Einfluss das Spiel mit Gleichaltrigen, auf die sozial-emotionale Entwicklung hat. Abschlie�end wurde auf die Frage eingegangen, inwieweit sich sozial-emotionale Kompetenzen durch Spiel und Bewegung f�rdern lassen. Hierzu gab es dann noch einen kleinen Ausblick auf das Projekt SEKIP, das unter der Leitung von Frau Prof. Dr. Zimmer, zu dem Thema der vorliegenden Arbeit, noch bis Juni 2014 l�uft.
  \end{flushleft}
  \begin{flushleft}
    Durch die Hausarbeit konnte gezeigt werden, welchen Stellenwert eine entsprechende sozial-emotionale Entwicklung f�r den Lebenslauf eines Menschen hat und wie wichtig dabei die Gesellschaft ist. Der Grundstein hierf�r wird in der fr�hen Kindheit gelegt. Daher ist es auch nicht verwunderlich, dass sich gerade die Kindergartenzeit anbietet, die Entwicklung der sozial-emotionalen Kompetenzen bei Kindern zu unterst�tzen und zu f�rdern. So bieten Spiele und Bewegung vielerlei M�glichkeiten, soziale Lernprozesse zu gestalten. Im Freispiel sowie durch gezielte Spiel- und Bewegungsangebote bekommen Kinder die Gelegenheit, Regeln des Sozialenverhaltens zu erproben.
  \end{flushleft}
  \begin{flushleft}
    Nun zu meinem pers�nlichen Fazit. Ich hatte mir von dieser Facharbeit erhofft, zwei Sachen zu erfahren. Zum einen war es mir wichtig zu erfahren, inwieweit das Thema in der Fach�ffentlichkeit thematisiert und diskutiert wird. Zum anderen wollte ich wissen, inwiefern zu meiner eingehenden Frage, wissenschaftliche Erkenntnisse existieren.
  \end{flushleft}
  \begin{flushleft}
    Leider wurden nicht beide Ziele zu meiner vollsten Zufriedenheit erf�llt. Ich hatte zwar im Rahmen dieser Arbeit die Gelegenheit, viele Erkenntnisse �ber die Meinungen der Fach�ffentlichkeit zu erlangen, da es mir hierbei in keiner Weise an Fachliteratur gemangelt hat. Dadurch habe ich neue Erkenntnisse �ber die Wichtigkeit von Spiel und Bewegung erlangt, die mir auch in meinem weiteren Berufsleben sehr hilfreich sein werden. Auch auf das neue Wissen, �ber soziale und emotionale Kompetenzen, werde ich in Zukunft zur�ckgreifen. 
    Was mich aber �berrascht hat  war, dass es keine ausreichenden wissenschaftlichen Studien zu diesem Thema gibt bzw. bisher noch keine vorliegen.
  \end{flushleft}
  \section{Ausblick}
    \begin{flushleft}
      Durch die im Rahmen dieser Hausarbeit durchgef�hrte Untersuchung, konnte ich erste Erkenntnisse zu den Auswirkungen von Spiel und Bewegung auf die sozial-emotionale Kompetenzen erbringen. Derzeitig fehlen noch ausreichende wissenschaftliche Erkenntnisse, um dem Thema vollends gerecht zu werden. Die SEKIP Studie befasst sich zurzeit zwar eingehend mit dem Thema dieser Arbeit, aber es liegen noch keine Ergebnisse vor, da das Projekt erst im Juni 2014 endet. Daher m�chte ich erst neue Ergebnisse und Erkenntnisse dieser Studie abwarten, bevor ich anhand dieser, neue Forschungsfragen zu diesem Thema verfolge und entwickele. Trotz allem hat sich f�r mich noch eine Weiterf�hrendefrage ergeben -- Inwieweit beeinflussen sozial-emotionale Kompetenzen den Schulerfolg von Kindern? Ich k�nnte mir gut vorstellen, diese Fragestellung in einer weiteren wissenschaftlichen Arbeit auszuarbeiten.
    \end{flushleft}