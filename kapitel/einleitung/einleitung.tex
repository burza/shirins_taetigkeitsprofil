%---------------------------------------------------------------------------------------------------
% Einf�hrung
%---------------------------------------------------------------------------------------------------
% \newpage 
%%\part{Anfang}
\chapter{Einleitung}

\begin{flushleft}
Kindertagesst�tten werden heute immer mehr zu Orten an denen Qualit�t und Vielfalt gro�geschrieben werden. Ihre Aufgabe wird immer mehr Kinder auf die wissensorientierte Gesellschaft vorzubereiten und die sozial-emotionalen und  lernmethodischen Kompetenzen zu f�rdern. Dabei ist es wichtig sie zu eigenverantwortlichen Wesen zu bilden und erziehen. Die ver�nderten Strukturen denen Kindertagesst�tten t�glich aufs Neue begegnen m�ssen, haben ebenfalls viel mit der Vielfalt an Kulturen und Lebenskonzepten zu tun. Ein konstruktiver Umgang mit diesen Ver�nderungen ist f�r Kinder, Eltern und das p�dagogische Fachpersonal von gro�er Bedeutung. Die Zusammenarbeit mit anderen Institutionen, kann hierbei sehr hilfreich sein.
\end{flushleft}

\begin{flushleft}
All dies h�rt sich nach sehr viel Arbeit an und man bekommt das unb�ndige Gef�hl man m�sse auf mehr als einer Hochzeit gleichzeitig tanzen. Damit das Zusammenspiel dieser verschiedenen Komponenten auch gelingt, bedarf es an guten Organisationsstrukturen und -prozessen. Die Grundlage bietet eine achtsame, vorausschauende und nachhaltige Leitung, die durch wohlwollende Kommunikation gepr�gt ist. Das hei�t eine Kita-Leitungskraft hat die Aufgabe ihr Team und ihr Unternehmen zu leiten und zu managen, dies verlangt ihr eine Vielzahl an Kompetenzen ab. 
\end{flushleft}

\newpage
\begin{flushleft}
Im Verlauf dieser Arbeit werde ich folgender Frage nachgehen: Was umfasst das T�tigkeitsprofil einer Kita-Leitung? Dieser werde ich an Hand von den Ergebnissen, die ich w�hrend des Interviews und des Shadowings mit Herrn Strau� (Kita-Leitungskraft der Kindertagesst�tte der Gemeinde Oststeinbek) erlangt habe, ausf�hren. Am Anfang dieser Arbeit werde ich das Arbeitsfeld sowie das Aufgabenfeld erl�utern. Hierbei wird deutlich,wie vielf�ltig und anspruchsvoll die Arbeit einer Kita-Leitung ist. Des Weiteren werde ich auf den gesetzlichen Rahmen eingehen, nach dem sich eine Kindertagesst�tte zu richten hat. Au�erdem gebe ich einen kleinen Einblick welche verschiedenen Managmentdimensionen mit diesem T�tigkeitsbereich einhergehen und welche Kompetenzen daf�r ben�tigt werden. Im letzten Kapitel reflektiere ich die Analyseergebnisse vor auf dem Hintergrund der Wichtigkeit von Studieninhalten f�r die Arbeit als Kita-Leitungskraft. Abschlie�end fasse ich zusammen und gebe mein Fazit wieder. Viele der Themen in die Arbeit konnte ich nur anrei�en, da es sonst den Umfang dieser Arbeit �berschritten h�tte. Trotz allem hoffe ich, dass diese Hausarbeit dem Leser einen kleinen Einblick in das T�tigkeitsprofil einer Kita-Leitung gibt.
\end{flushleft}

\newpage